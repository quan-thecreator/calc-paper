\documentclass{report}
\usepackage{cancel}
\usepackage{rotating}
\input{preamble}
\input{macros}
\input{letterfonts}
\title{\Huge{Calculuus - The Reckoning}}
\author{\huge{Krishna ;)}}
\date{}
\begin{document}

\maketitle
\newpage% or \cleardoublepage
% \pdfbookmark[<level>]{<title>}{<dest>}
\pdfbookmark[section]{\contentsname}{toc}
\tableofcontents
\pagebreak

\chapter{}
\section{The Tests}


\dfn{The Geometric Series Test}{
	Suppose you have a series $$
	\sum_{n=1}^{\infty}{a(r)^{n-1}}
	$$
	or any other series with a common ratio $r$ and an initial $a$, involving a multiplication of that variate value $r$ over the indexes of said infinite series $\sum{a_n}$. The series converges if:
	$$r<1$$
	\text{and diverges when:}
	$$r>1$$
	If the infinite series $\sum{a_n}$ converges the sum can be found with:
	$$S=\frac{a}{1-r}$$
	And the interval of converges can also be found if $r$ is a variate quantity with the inequality $|r|<1$. Using this method, the \textit{endpoints} must be tested to see if the bounds are inclusive of exclusive.
}
\dfn{Telescoping Series Test}{
	Suppose you have a series $\sum{a_n}$ of the form$$
		a_n = b_n - b_{n+1}
	$$ where 
	$$
		\sum{a_n} =  \sum_{n=1}^{\infty}{b_n - b_{n+1}} = (b_1-\cancel{b_2}) + (\cancel{b_2}-\cancel{b_3}) + (\cancel{b_3}-b_4) + ...
	$$
	$$
		\implies S=b_1 \therefore S_k=b_1-b_{k+1}
	$$
}
\dfn{$n^{\mathrm{th}}$ term test}{
\text{Suppose you have a series} $\sum{a_n}$
	\begin{align*}
		L = \lim_{n\to\infty}{a_n}
	\end{align}
	$\sum{a_n}$ diverges if:
	\begin{align*}
		L\ne0
	\end{align}
}
\begin{note}
This test only tests for divergence.
\end{note}
\dfn{The Integral Test}{
	Suppose you have a series $\sum_{n=j}^{\infty}{a_n}$, where $a_n = f(x)$, and $f(x)$ is continuous, positive and decreasing on the interval $[j,\infty)$.
	\begin{align*}
		\sum_{n=j}^{\infty}{a_n} \text{ converges if } \\
		L= \int_{j}^{\infty}{f(x)} = \lim_{b\to\infty}{\int_{j}^b{f(x)}} \\
		\text{where $L$ is some positive, finite value meeting the condition: } 0<L<\infty \\
		\text{The sum diverges if $L$ diverges (DNE counts)}
	\end{align}
}
\dfn{p-series Test}{
	Suppose you have a series of the form:
	$$\sum_{n=1}^\infty{\frac{1}{n^p}}$$
	This series converges if $p>1$ and diverges if $0<p\le1$
}
\dfn{The Alternating Series Test}{
	Suppose you have a series:
	\begin{align*}
		\sum{a_n}=\sum_{n=1}^\infty{(-1)^{n+k}b_n}\\
		||a_n||=b_n
	\end{align}
	If:
	\begin{align*}
		\lim_{n\to\infty}b_n=0\\
		b_n>b_{n+1}
	\end{align}
	The series $\sum{a_n}$ converges.
}
\begin{note}
This test only tests for convergence.
\end{note}
\dfn{The Ratio Test}{
	Suppose you have a series $\sum{a_n}$ 
	$$
	L = \lim_{n\to\infty}{\frac{a_{n+1}}{a_n}}
	$$
	The series converges if:
	$$L<1$$
	Diverges if:
	$$L>1$$
	and fails when:
	$$L=1$$
}
\begin{note}
This test is especially useful when dealing with factorials and exponential, as the powers reduce. There is the off-chance that you find a product such as $2*5*7*9*(2n-1)$, where this technique tends to work well, just as it does in the factorial pattern.
\end{note}
\pagebreak
\section{The Tests of Character}

\end{document}


