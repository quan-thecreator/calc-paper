\documentclass{report}
\usepackage{cancel}
\usepackage{rotating}
\input{preamble}
\input{macros}
\input{letterfonts}
\title{\Huge{Calculuus - The Reckoning}}
\author{\huge{Krishna ;)}}
\date{}
\begin{document}

\maketitle
\newpage% or \cleardoublepage
% \pdfbookmark[<level>]{<title>}{<dest>}
\pdfbookmark[section]{\contentsname}{toc}
\tableofcontents
\pagebreak

\chapter{}
\section{The Tests}


\dfn{The Geometric Series Test}{
	Suppose you have a series $$
	\sum_{n=1}^{\infty}{a(r)^{n-1}}
	$$
	or any other series with a common ratio $r$ and an initial $a$, involving a multiplication of that variate value $r$ over the indexes of said infinite series $\sum{a_n}$. The series converges if:
	$$r<1$$
	\text{and diverges when:}
	$$r>1$$
	If the infinite series $\sum{a_n}$ converges the sum can be found with:
	$$S=\frac{a}{1-r}$$
	And the interval of converges can also be found if $r$ is a variate quantity with the inequality $|r|<1$. Using this method, the \textit{endpoints} must be tested to see if the bounds are inclusive of exclusive.
}
\dfn{Telescoping Series Test}{
	Suppose you have a series $\sum{a_n}$ of the form$$
		a_n = b_n - b_{n+1}
	$$ where 
	$$
		\sum{a_n} =  \sum_{n=1}^{\infty}{b_n - b_{n+1}} = (b_1-\cancel{b_2}) + (\cancel{b_2}-\cancel{b_3}) + (\cancel{b_3}-b_4) + ...
	$$
	$$
		\implies S=b_1 \therefore S_k=b_1-b_{k+1}
	$$
}
\dfn{$n^{\mathrm{th}}$ term test}{
\text{Suppose you have a series} $\sum{a_n}$
	\begin{align*}
		L = \lim_{n\to\infty}{a_n}
	\end{align}
	$\sum{a_n}$ diverges if:
	\begin{align*}
		L\ne0
	\end{align}
}
\begin{note}
This test only tests for divergence.
\end{note}
\dfn{The Integral Test}{
	Suppose you have a series $\sum_{n=j}^{\infty}{a_n}$, where $a_n = f(x)$, and $f(x)$ is continuous, positive and decreasing on the interval $[j,\infty)$.
	\begin{align*}
		\sum_{n=j}^{\infty}{a_n} \text{ converges if } \\
		L= \int_{j}^{\infty}{f(x)} = \lim_{b\to\infty}{\int_{j}^b{f(x)}} \\
		\text{where $L$ is some positive, finite value meeting the condition: } 0<L<\infty \\
		\text{The sum diverges if $L$ diverges (DNE counts)}
	\end{align}
}
\dfn{p-series Test}{
	Suppose you have a series of the form:
	$$\sum_{n=1}^\infty{\frac{1}{n^p}}$$
	This series converges if $p>1$ and diverges if $0<p\le1$
}
\dfn{The Alternating Series Test}{
	Suppose you have a series:
	\begin{align*}
		\sum{a_n}=\sum_{n=1}^\infty{(-1)^{n+k}b_n}\\
		|a_n|=b_n
	\end{align}
	If:
	\begin{align*}
		\lim_{n\to\infty}b_n=0\\
		b_n>b_{n+1}
	\end{align}
	The series $\sum{a_n}$ converges.
}
\begin{note}
This test only tests for convergence.
\end{note}
\dfn{The Ratio Test}{
	Suppose you have a series $\sum{a_n}$ 
	$$
	L = \lim_{n\to\infty}{|\frac{a_{n+1}}{a_n}|}
	$$
	The series converges if:
	$$L<1$$
	Diverges if:
	$$L>1$$
	and fails when:
	$$L=1$$
}
\begin{note}
This test is especially useful when dealing with factorials and exponential, as the powers reduce. There is the off-chance that you find a product such as $2*5*7*9*(2n-1)$, where this technique tends to work well, just as it does in the factorial pattern.
\end{note}
\dfn{Direct Comparison Test}{
	Suppose you have a series $\sum{a_n}$ and find a series $\sum{b_n}$. $$a_n,b_n\ge 0,n \in \mathbb{R}$$ In order to prove $\sum{a_n}$ converges you must show that:
	$$a_n\le b_n$$ for all $n$, and that $\sum{b_n}$ converges using any other test of your choosing. In order to prove $\sum{a_n}$ diverges you must show that:
	$$a_n\ge b_n$$ for all $n$, and that $\sum{b_n}$ diverges.
}
\begin{note}
This test works well for compositions of rationals and exponential, where $\sum{b_n}$ is simply $\sum{a_n}$ without a constant in the denominator. Try to use it sparingly and only when the opportunity presents itself.
\end{note}
\dfn{Limit Comparison Test}{
	Suppose you have a series $\sum{a_n}$, and you find a series $\sum{b_n}$ $$a_n\ge0,b_n>0,n \in \mathbb{R}$$
	$$c=\lim_{n\to\infty}{\frac{a_n}{b_n}}$$
	$$0<c<\infty$$
	$\sum{a_n}$ converges if $\sum{b_n}$ converges, $\sum{a_n}$ diverges if $\sum{b_n}$ diverges.
}
\dfn{Root Test}{
	Suppose you have a series $\sum{a_n}$
	$$L=\lim_{n\to\infty}{\sqrt[n]{|a_n|}}$$
	Then 1 of 3 cases is true:

	1. if $L<1$ $\sum{a_n}$ is absolutely convergent (hence convergent)

	2. if $L>1$ $\sum{a_n}$ is divergent
	
	3. if $L=1$ the test fails
}
\dfn{Absolute vs. Conditional Convergence}{
	\textbf{This test only matters if ${a_n}$ is variate.}

	$\sum{a_n}$ is absolutely convergent if $\sum{|a_n|}$ converges. $\sum{a_n}$ is conditionally convergent if $\sum{|a_n|}$ diverges.
}
\clm{Interval of Convergence}{}{
	Test endpoints any time when finding the interval of convergence after using the Root Test, Ratio Test, Geometric Series Test, or (will almost never be asked) the Alternating Series Test. With the alternating series test, it is important to watch if $$b_n>b_{n+1}$$
}
\dfn{The Alternating Series Error Bound}{
	$\mathrm{Error} \le |a_{n+1}|$, simply reason this out, and you can prove it if you really feel the need to.
}
\pagebreak
\section{The Tests of Character}
\qs{}{
	Does the following series converge or diverge? Specify the test, show all steps, and identify the correct radius of convergence and interval of convergence with respect to $x$.
	$$\sum_{n=1}^{\infty}{\frac{(-1)^{n+1}(x-4)^{n}}{n9^n}}$$
	\begin{turn}{180} 
	{\textbf{Answer.}
	Use the ratio test and find: $R=9$, and interval of convergence to be $(-5,13]$
	\end{turn}
	}
}
For the following 4 problems use: $$f(x)=\sum_{n=1}^{\infty}{(\tfrac{x}{3})^n}$$
\qs{}{
	Find the interval of convergence for $f(x)$, if the series converges. Identify the test.
	\begin{turn}{180}
	{
	\textbf{Answer.} Use Geometric Series test to identify that $R=3$ and interval of convergence to be $(-3,3)$
	}
	\end{turn}
}
\qs{}{
	Find the interval of convergence for $f'(x)$, if the series converges. Identify the test.
	\begin{turn}{180}
	{
	\textbf{Answer.} Use Ratio test to identify that $R=3$ and interval of convergence to be $(-3,3)$
	}
	\end{turn}
}
\qs{}{
	Find the interval of convergence for $f''(x)$, if the series converges. Identify the test.
	\begin{turn}{180}
	{
	\textbf{Answer.} Use Ratio test to identify that $R=3$ and interval of convergence to be $(-3,3)$
	}
	\end{turn}
}
\qs{}{
	Find the interval of convergence for $\int{f(x)\mathrm{d}x}$, if the series converges. Identify the test.
	\begin{turn}{180}
	{
	\textbf{Answer.} Use Ratio test to identify that $R=3$ and interval of convergence to be $[-3,3)$
	}
	\end{turn}
}
\begin{turn}{180}
\begin{note}
When deriving a series, the bound may only loose convergence, not gain. The opposite is true for integration. If you receive another series of questions like this you can simply use the same bound after setting up the test, and only test the endpoints when integrating, not deriving as that would only waste time.
\end{note}
\end{turn}
For the following 5 problems use $$f(x)=\sum_{n=1}^{\infty}{\frac{(-1^{n+1}(x-1)^{n+1})}{n+1}}$$
\qs{}{Compute $f(1),f'(x),f''(x),\int{f(x)\mathrm{d}x}$\newline
\begin{turn}{180}
{
$f(1)=0$,$f'(x)=\sum_{n=1}^{\infty}{\frac{(-1)^{n+1}}{(n+1)^2}(x-1)^n},f''(x)=\sum_{n=1}^{\infty}{\frac{(-1)^{n+1}(x-1)^{n-1}}{(n+1)^2n}}$ \newline 
$\int{f(x)\mathrm{d}x}=\sum_{n=1}^{\infty}{\frac{(-1)^{n+1}(x-1)^{n+2}}{(n+1)(n+2)}}$
}
\end{turn}
}
\qs{}{
	Find the interval of convergence for $f(x)$, if the series converges. Identify the test.
	\begin{turn}{180}
	{
	\textbf{Answer.} Use Ratio Series test to identify that $R=1$ and interval of convergence to be $(0,2]$
	}
	\end{turn}
}
\qs{}{
	Find the interval of convergence for $f'(x)$, if the series converges. Identify the test.
	\begin{turn}{180}
	{
	\textbf{Answer.} Use Ratio test to identify that $R=1$ and interval of convergence to be $(0,2)$
	}
	\end{turn}
}
\qs{}{
	Find the interval of convergence for $f''(x)$, if the series converges. Identify the test.
	\begin{turn}{180}
	{
	\textbf{Answer.} Use Ratio test to identify that $R=1$ and interval of convergence to be $(0,2)$
	}
	\end{turn}
}
\qs{}{
	Find the interval of convergence for $\int{f(x)\mathrm{d}x}$, if the series converges. Identify the test.
	\begin{turn}{180}
	{
	\textbf{Answer.} Use Limit Comparison Test test to identify that $R=1$ and interval of convergence to be $[0,2]$
	}
	\end{turn}
}

\end{document}
